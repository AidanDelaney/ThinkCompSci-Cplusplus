% LaTeX source for textbook ``How to think like a computer scientist''
% Copyright (C) 1999  Allen B. Downey

% This LaTeX source is free software; you can redistribute it and/or
% modify it under the terms of the GNU General Public License as
% published by the Free Software Foundation (version 2).

% This LaTeX source is distributed in the hope that it will be useful,
% but WITHOUT ANY WARRANTY; without even the implied warranty of
% MERCHANTABILITY or FITNESS FOR A PARTICULAR PURPOSE.  See the GNU
% General Public License for more details.

% Compiling this LaTeX source has the effect of generating
% a device-independent representation of a textbook, which
% can be converted to other formats and printed.  All intermediate
% representations (including DVI and Postscript), and all printed
% copies of the textbook are also covered by the GNU General
% Public License.

% This distribution includes a file named COPYING that contains the text
% of the GNU General Public License.  If it is missing, you can obtain
% it from www.gnu.org or by writing to the Free Software Foundation,
% Inc., 59 Temple Place - Suite 330, Boston, MA 02111-1307, USA.


\chapter{More structures}
\label{time}
\index{struct}

\section{Time}
\index{struct!Time}
\index{Time}

As a second example of a user-defined structure, we will define a type
called {\tt Time}, which is used to record the time of day.  The
various pieces of information that form a time are the hour, minute
and second, so these will be the instance variables of the structure.

The first step is to decide what type each instance variable should
be.  It seems clear that {\tt hour} and {\tt minute} should be
integers.  Just to keep things interesting, let's make {\tt second} a
{\tt double}, so we can record fractions of a second.

Here's what the structure definition looks like:

\begin{lstlisting}
struct Time {
  int hour, minute;
  double second;
};
\end{lstlisting}
%
We can create a {\tt Time} object in the usual way:

\begin{lstlisting}
  Time time = { 11, 59, 3.14159 };
\end{lstlisting}
%
The state diagram for this object looks like this:

\vspace{0.1in}
\centerline{\epsfig{figure=time.eps}}
\vspace{0.1in}

The word ``instance'' is sometimes used when we talk about objects,
because every object is an instance (or example) of some type.  The
reason that instance variables are so-named is that every instance of
a type has a copy of the instance variables for that type.

\section{{\tt printTime}}
\label{printobject}
\index{output}
\index{statement!output}
\index{object!output}

When we define a new type it is a good idea to write
function that displays the instance variables in a human-readable
form.  For example:

\begin{lstlisting}
void printTime (Time& t) {
  cout << t.hour << ":" << t.minute << ":" << t.second << endl;
}
\end{lstlisting}
%
The output of this function, if we pass {\tt time}
an argument, is {\tt 11:59:3.14159}.

\begin{lstlisting}
#include <iostream>

using namespace std;

struct Time {
  int hour, minute;
  double second;
};


void printTime (Time& t) {
  cout << t.hour << ":" << t.minute << ":" << t.second << endl;
  cout << "Time is " << t.hour << " hour " << t.minute << " minutes " << t.second << "  seconds  "<<endl;
}


int main ()
{
 Time time = { 11, 59, 3.14159 };
 printTime(time);
 
 return 0;
}
\end{lstlisting}
%
\section{Functions for objects}
\label{objectops}
\index{object}
\index{function!for objects}

In the next few
sections, I will demonstrate several possible interfaces for
functions that operate on objects.  For some operations, you will have a
choice of several possible interfaces, so you should consider the pros
and cons of each of these:

\begin{description}

\item[pure function:]  Takes objects and/or basic types as
arguments but does not modify the objects.  The return value is
either a basic type or a new object created inside the function.

\item[modifier:]  Takes objects as parameters and modifies some
or all of them.  Often returns void. \index{void}

\item[fill-in function:]  One of the parameters is an ``empty''
object that gets filled in by the function.  Technically, this is
a type of modifier.

\end{description}

\section{Pure functions}
\index{pure function}
\index{function}
\index{function!pure function}

A function is considered a pure function if the result depends only on
the arguments, and it has no side effects like modifying an argument
or outputting something.  The only result of calling a pure function is
the return value.

One example is {\tt after}, which compares two {\tt Time}s and
returns a {\tt bool} that indicates whether the first operand
comes after the second:

\begin{lstlisting}
bool after (Time& time1, Time& time2) {
  if (time1.hour > time2.hour) return true;
  if (time1.hour < time2.hour) return false;

  if (time1.minute > time2.minute) return true;
  if (time1.minute < time2.minute) return false;

  if (time1.second > time2.second) return true;
  return false;
}
\end{lstlisting}
%
What is the result of this function if the two times are equal?  Does
that seem like the appropriate result for this function?  If you were
writing the documentation for this function, would you mention that case
specifically?

A second example is {\tt addTime}, which calculates the sum of two
times.  For example, if it is {\tt 9:14:30}, and your breadmaker takes
3 hours and 35 minutes, you could use {\tt addTime} to figure out when
the bread will be done.

Here is a rough draft of this function that is not quite right:

\begin{lstlisting}
Time addTime (Time& t1, Time& t2) {
  Time sum;
  sum.hour = t1.hour + t2.hour;
  sum.minute = t1.minute + t2.minute;
  sum.second = t1.second + t2.second;
  return sum;
}
\end{lstlisting}
%
Here is an example of how to use this function.  If {\tt currentTime}
contains the current time and {\tt breadTime} contains the amount
of time it takes for your breadmaker to make bread, then you
could use {\tt addTime} to figure out when the bread will be
done.

\begin{lstlisting}
  Time currentTime = { 9, 14, 30.0 };
  Time breadTime = { 3, 35, 0.0 };
  Time doneTime = addTime (currentTime, breadTime);
  printTime (doneTime);
\end{lstlisting}
%
The output of this program is {\tt 12:49:30}, which is
correct.  On the other hand, there are cases where the result
is not correct.  Can you think of one?

The problem is that this function does not deal with cases
where the number of seconds or minutes adds up to more than
60.  When that happens we have to ``carry'' the extra seconds
into the minutes column, or extra minutes into the hours
column.

Here's a second, corrected version of this function.

\begin{lstlisting}
Time addTime (Time& t1, Time& t2) {
  Time sum;
  sum.hour = t1.hour + t2.hour;
  sum.minute = t1.minute + t2.minute;
  sum.second = t1.second + t2.second;

  if (sum.second >= 60.0) {
    sum.second -= 60.0;
    sum.minute += 1;
  }
  if (sum.minute >= 60) {
    sum.minute -= 60;
    sum.hour += 1;
  }
  return sum;
}
\end{lstlisting}
%
Although it's correct, it's starting to get big.  Later,
I will suggest an alternate approach to this problem that
will be much shorter.

\index{increment}
\index{decrement}
\index{operator!increment}
\index{operator!decrement}

This code demonstrates two operators we have not seen before, {\tt +=}
and {\tt -=}.  These operators provide a concise way to increment and
decrement variables.  For example, the statement {\tt sum.second -=
60.0;} is equivalent to {\tt sum.second = sum.second - 60;}

\section{{\tt const} parameters}

You might have noticed that the parameters for {\tt after}
and {\tt addTime} are being passed by reference.  Since
these are pure functions, they do not modify the parameters
they receive, so I could just as well have passed them by
value.

The advantage of passing by value is that the calling function
and the callee are appropriately encapsulated---it is not possible
for a change in one to affect the other, except by affecting
the return value.

On the other hand, passing by reference usually is more efficient,
because it avoids copying the argument.  Furthermore, there is a nice
feature in C++, called {\tt const}, that can make reference parameters
just as safe as value parameters.

If you are writing a function and you do not intend to modify
a parameter, you can declare that it is a {\bf constant
reference parameter}.  The syntax looks like this:

\begin{lstlisting}
void printTime (const Time& time) ...
Time addTime (const Time& t1, const Time& t2) ...
\end{lstlisting}
%
I've included only the first line of the functions.  If you tell
the compiler that you don't intend to change a
parameter, it can help remind you.  If you try to change one,
you should get a compiler error, or at least a warning.

\index{run-time error}
\index{error!run-time}

\section{Modifiers}
\index{modifier}
\index{function!modifier}

Of course, sometimes you {\em want} to modify one of the
arguments.  Functions that do are called modifiers.

As an example of a modifier, consider {\tt increment},
which adds a given number of seconds to a {\tt Time} object.
Again, a rough draft of this function looks like:

\begin{lstlisting}
void increment (Time& time, double secs) {
  time.second += secs;

  if (time.second >= 60.0) {
    time.second -= 60.0;
    time.minute += 1;
  }
  if (time.minute >= 60) {
    time.minute -= 60;
    time.hour += 1;
  }
}
\end{lstlisting}
%
The first line performs the basic operation; the remainder
deals with the special cases we saw before.

Is this function correct?  What happens if the argument {\tt secs}
is much greater than 60?  In that case, it is not enough to
subtract 60 once; we have to keep doing it until {\tt second}
is below 60.  We can do that by replacing the {\tt if}
statements with {\tt while} statements:

\begin{lstlisting}
void increment (Time& time, double secs) {
  time.second += secs;

  while (time.second >= 60.0) {
    time.second -= 60.0;
    time.minute += 1;
  }
  while (time.minute >= 60) {
    time.minute -= 60;
    time.hour += 1;
  }
}
\end{lstlisting}
%
This solution is correct, but not very efficient.
Can you think of a solution that does not require iteration?

\section{Fill-in functions}
\index{fill-in function}
\index{function!fill-in}

Occasionally you will see functions like {\tt addTime} written
with a different interface (different arguments and return values).
Instead of creating a new object every time {\tt addTime} is
called, we could require the caller to provide an ``empty''
object where {\tt addTime} can store the result.  Compare
the following with the previous version:

\begin{lstlisting}
void addTimeFill (const Time& t1, const Time& t2, Time& sum) {
  sum.hour = t1.hour + t2.hour;
  sum.minute = t1.minute + t2.minute;
  sum.second = t1.second + t2.second;

  if (sum.second >= 60.0) {
    sum.second -= 60.0;
    sum.minute += 1;
  }
  if (sum.minute >= 60) {
    sum.minute -= 60;
    sum.hour += 1;
  }
}
\end{lstlisting}
%
One advantage of this approach is that the caller has the option of
reusing the same object repeatedly to perform a series of additions.
This can be slightly more efficient, although it can be confusing
enough to cause subtle errors.  For the vast majority of programming,
it is worth a spending a little run time to avoid a lot of debugging
time.

Notice that the first two parameters can be declared {\tt const},
but the third cannot.

\section{Which is best?}
\index{programming style}

Anything that can be done with modifiers and fill-in functions can also
be done with pure functions.  In fact, there are programming
languages, called {\bf functional} programming languages, that only
allow pure functions.  Some programmers believe that programs that use
pure functions are faster to develop and less error-prone than
programs that use modifiers.  Nevertheless, there are times when
modifiers are convenient, and cases where functional programs
are less efficient.

In general, I recommend that you write pure functions whenever
it is reasonable to do so, and resort to modifiers only if there
is a compelling advantage.  This approach might be called a
functional programming style.

\section{Incremental development versus planning}
\index{incremental development}
\index{prototyping}
\index{program development!incremental}
\index{program development!planning}

In this chapter I have demonstrated an approach to program
development I refer to as {\bf rapid prototyping with iterative
improvement}.  In each case, I wrote a rough draft (or prototype)
that performed the basic calculation, and then tested it on
a few cases, correcting flaws as I found them.

Although this approach can be effective, it can lead to code
that is unnecessarily complicated---since it deals with many
special cases---and unreliable---since it is hard to know if
you have found all the errors.

An alternative is high-level planning, in which a little insight
into the problem can make the programming much easier.  In
this case the insight is that a {\tt Time} is really a three-digit
number in base 60!  The {\tt second} is the ``ones column,''
the {\tt minute} is the ``60's column'', and the {\tt hour}
is the ``3600's column.''

When we wrote {\tt addTime} and {\tt increment}, we were effectively
doing addition in base 60, which is why we had to ``carry'' from one
column to the next.

\index{arithmetic!base 60}
\index{arithmetic!floating-point}

Thus an alternate approach to the whole problem is to convert
{\tt Time}s into {\tt double}s and take advantage of the fact that
the computer already knows how to do arithmetic with {\tt double}s.
Here is a function that converts a {\tt Time} into a {\tt double}:

\begin{lstlisting}
double convertToSeconds (const Time& t) {
  int minutes = t.hour * 60 + t.minute;
  double seconds = minutes * 60 + t.second;
  return seconds;
}
\end{lstlisting}
%
Now all we need is a way to convert from a {\tt double}
to a {\tt Time} object:

\begin{lstlisting}
Time makeTime (double secs) {
  Time time;
  time.hour = int (secs / 3600.0);
  secs -= time.hour * 3600.0;
  time.minute = int (secs / 60.0);
  secs -= time.minute * 60;
  time.second = secs;
  return time;
}
\end{lstlisting}
%
You might have to think a bit to convince yourself that the technique
I am using to convert from one base to another is correct.  Assuming
you are convinced, we can use these functions to rewrite {\tt addTime}:

\begin{lstlisting}
Time addTime (const Time& t1, const Time& t2) {
  double seconds = convertToSeconds (t1) + convertToSeconds (t2);
  return makeTime (seconds);
}
\end{lstlisting}
%
This is much shorter than the original version, and it is much easier
to demonstrate that it is correct (assuming, as usual, that the
functions it calls are correct).  As an exercise, rewrite {\tt
increment} the same way.


\section{Generalization}
\index{generalization}

In some ways converting from base 60 to base 10 and back is
harder than just dealing with times.  Base conversion is more
abstract; our intuition for dealing with times is better.

But if we have the insight to treat times as base 60 numbers,
and make the investment of writing the conversion functions
({\tt convertToSeconds} and {\tt makeTime}), we get
a program that is shorter, easier to read and debug, and more
reliable.

It is also easier to add more features later.  For example, imagine
subtracting two {\tt Time}s to find the duration between them.  The
naive approach would be to implement subtraction with
borrowing.  Using the conversion functions would be easier and more
likely to be correct.

Ironically, sometimes making a problem harder (more general)
makes is easier (fewer special cases, fewer opportunities for error).

\section{Algorithms}
\label{algorithm}
\index{algorithm}

When you write a general solution for a class of problems, as opposed
to a specific solution to a single problem, you have written an {\bf
algorithm}.  I mentioned this word in Chapter 1, but did not define it
carefully.  It is not easy to define, so I will try a couple of
approaches.

First, consider something that is not an algorithm.
When you learned to multiply single-digit numbers, you probably
memorized the multiplication table.  In effect, you memorized 100
specific solutions.  That kind of knowledge is not really algorithmic.

But if you were ``lazy,'' you probably cheated by learning a few
tricks.  For example, to find the product of $n$ and 9, you can write
$n-1$ as the first digit and $10-n$ as the second digit.  This trick
is a general solution for multiplying any single-digit number by 9.
That's an algorithm!

Similarly, the techniques you learned for addition with carrying,
subtraction with borrowing, and long division are all algorithms.  One
of the characteristics of algorithms is that they do not require any
intelligence to carry out.  They are mechanical processes in which
each step follows from the last according to a simple set of rules.

In my opinion, it is embarrassing that humans spend so much time in
school learning to execute algorithms that, quite literally, require
no intelligence.

On the other hand, the process of designing algorithms is interesting,
intellectually challenging, and a central part of what we call
programming.

Some of the things that people do naturally, without difficulty
or conscious thought, are the most difficult to express
algorithmically.  Understanding natural language is a good
example.  We all do it, but so far no one has been able to
explain {\em how} we do it, at least not in the form of an
algorithm.

Later in this book, you will have the opportunity to design
simple algorithms for a variety of problems.  If you take
the next class in the Computer Science sequence, Data Structures,
you will see some of the most interesting, clever, and
useful algorithms computer science has produced.

\section{Glossary}

\begin{description}

\item[instance:]  An example from a category.  My cat is an
instance of the category ``feline things.''  Every object is
an instance of some type.

\item[instance variable:]  One of the named data items that make
up an structure.  Each structure has its own copy of
the instance variables for its type.

\item[constant reference parameter:]  A parameter that is passed
by reference but that cannot be modified.

\item[pure function:]  A function whose result depends only on its
parameters, and that has so effects other than returning
a value.

\item[functional programming style:]  A style of program design
in which the majority of functions are pure.

\item[modifier:]  A function that changes one or more of the objects
it receives as parameters, and usually returns {\tt void}.

\item[fill-in function:]  A function that takes an ``empty''
object as a parameter and fills it its instance variables instead
of generating a return value.

\item[algorithm:]  A set of instructions for solving a class of
problems by a mechanical, unintelligent process.

\index{instance}
\index{instance variable}
\index{pure function}
\index{functional programming}
\index{modifier}
\index{algorithm}

\end{description}

